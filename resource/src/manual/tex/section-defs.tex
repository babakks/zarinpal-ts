\section{تعاریف}
\label{بخش:تعاریف}
تعاریف و اصطلاحات به کار رفته در این نوشتار
به شرح زیر هستند:
\begin{itemize}
	
\فقره \متن‌سیاه{پرداخت الکترونیکی}\\
فرآیند انتقال وجه از یک حساب بانکی به حساب
مربوط به سامانهٔ گیرنده.

\فقره \متن‌سیاه{کاربر انسانی}\\
کاربر انسانی که قصد دارد مبلغی را در ازای
دریافت کالا یا خدمات پرداخت کند.

\فقره \متن‌سیاه{سامانهٔ گیرنده} (یا \متن‌سیاه{سامانهٔ مشتری})\\
سامانه‌ای که دریافت کنندهٔ وجه است؛
برای مثال وب‌سایت یک فروشگاه یا مؤسسهٔ خیریه. 
از آنجا که این سامانه از خدمات
\متن‌ایتالیک{سامانهٔ واسط} برای انجام
پرداخت‌های الکترونیکی استفاده می‌کند،
آن را \متن‌ایتالیک{سامانهٔ مشتری} نیز می‌نامیم.

\فقره \متن‌سیاه{سامانهٔ واسط}\\
سامانه‌ای که امکان پرداخت الکترونیکی از طریق
درگاه‌های بانکی یا دیگر درگاه‌های معتبر را در
اختیار دیگر سامانه‌ها، مثل فروشگاه‌ها، قرار می‌دهد.

	\begin{itemize}	
		\فقره \متن‌سیاه{زرین‌پال}\\
		یک سامانهٔ واسط که نوشتار حاضر
		در شرح استفاده از آن نوشته شده است.
	\end{itemize}

\فقره \متن‌سیاه{درگاه پرداخت}\\
درگاه اینترنتی مربوط به یک بانک یا سرویس‌دهندهٔ
معتبر دیگر که اسباب لازم برای انتقال وجه از
حساب کاربر انسانی به یک حساب مقصد را فراهم می‌کند.

\فقره \متن‌سیاه{مسیر بازگشت}\پانویس{Callback}\\
مسیری که کاربر انسانی پس از پایان تراکنش باید
به آن منتقل شود تا سامانهٔ گیرنده بتواند نتیجهٔ
عملیات را دریافت کند.

\end{itemize}
